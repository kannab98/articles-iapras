../preamble/preamble.tex
\title{Черновик статьи \\ \textbf{Моделирование схемы измерения DPR}}
\author{Понур К.А.}
\begin{document}
\maketitle
%\tableofcontents



%\section{Моделирование морского волнения}%
%\label{sec:modelirovanie_morskogo_volneniia}


%\subsection{Вычисление корреляционной функции}%
%\label{sub:vychislenie_korreliatsionnoi_funktsii}


%Для проверки качества моделирования, а также вычисления нового спектра волнения 
%необходимо вычислять функцию корреляции для модельной поверхности.

%По определению, 
%\begin{equation}
    %K_{\zeta}(\rho) = \mean{ \zeta(r) \zeta(r+\rho) }
%\end{equation}

%Для стационарного и эргодического процесса
%\begin{equation}
    %\label{eq:defK:2}
    %K_{\zeta}(\rho) = \frac{1}{r_0} \int\limits_{0}^{r_0}  \zeta(r) \zeta(r+\rho)
    %\dd r
%\end{equation}
%где $r_0$ должно быть велико по сравнению с чем-то.

%Расчет корреляции можно ускорить, используя теорему о корреляции, которая
%обычно формулируется следующим образом:
%\begin{equation}
    %K_{\zeta}(\rho) = \ifft{\fft*{\xi} \cdot \fft{\xi}} = 
    %\ifft{\abs{\fft{\xi}}^2}
%\end{equation}

%При помощи теоремы котельникова можно пока

%\begin{equation}
    %\label{eq:}
    %\Delta x = \frac{\pi}{k}
%\end{equation}

Расчет корреляции можно ускорить, используя теорему о корреляции, которая
обычно формулируется следующим образом:

\begin{equation}
    K_{\zeta}(\rho) = \ifft{\fft*{\zeta} \cdot \fft{\zeta}} = 
    \ifft{S(\kappa)}
\end{equation}

\begin{equation}
    \fft{\zeta} = \int\limits_{-\infty}^{\infty} \zeta(x_0,t) e^{-i kx_0} \dd x_0 
\end{equation}

В случае с моделью заостренной поверхности
\begin{equation}
    \fft{\zeta} = \int\limits_{-\infty}^{\infty} \zeta(x,t) e^{-i kx_0} \dd x,
\end{equation}


где 
\begin{equation}
    \dv{x}{x_0} = 1 -  \int\limits_{-\infty}^{\infty} k
    \fft{\zeta(x,t)}e^{-ikx_0} \dd x_0
\end{equation}

\begin{gather}
    \fft{\zeta} = 
    \int\limits_{-\infty}^{\infty} \zeta(x,t) e^{-i kx_0}
    \qty( 1 -  \int\limits_{-\infty}^{\infty} k \fft{\zeta(x_0,t)}e^{-ikx_0}\dd
    \kappa)
\dd x_0 \end{gather}

%\begin{equation}
    %K_{\sigma}(\rho) = \ifft{\kappa^2 S(\kappa)}
%\end{equation}

%\begin{equation}
    %K_{v}(\rho) = \ifft{\omega^2(\kappa) S(\kappa)}
%\end{equation}


%Раскроем обратное преобразование Фурье
%\begin{equation}
    %K_{\zeta}(\rho) = \frac{1}{T}\int\limits_{0}^{T} S(\kappa) e^{i\kappa\rho}
    %\dd \kappa
%\end{equation}








\end{document}

