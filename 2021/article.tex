../preamble/preamble.tex
\title{Черновик статьи \\ \textbf{Моделирование схемы измерения DPR}}
\author{Понур К.А.}
\begin{document}
\maketitle
\tableofcontents



\section{Моделирование морского волнения}%
\label{sec:modelirovanie_morskogo_volneniia}


\subsection{Вычисление корреляционной функции}%
\label{sub:vychislenie_korreliatsionnoi_funktsii}


Для проверки качества моделирования, а также вычисления нового спектра волнения 
необходимо вычислять функцию корреляции для модельной поверхности.

По определению, 
\begin{equation}
    K_{\zeta}(\rho) = \mean{ \zeta(r) \zeta(r+\rho) }
\end{equation}

Для стационарного и эргодического процесса
\begin{equation}
    \label{eq:defK:2}
    K_{\zeta}(\rho) = \frac{1}{r_0} \int\limits_{0}^{r_0}  \zeta(r) \zeta(r+\rho)
    \dd r
\end{equation}
где $r_0$ должно быть велико по сравнению с чем-то.

Расчет корреляции можно ускорить, используя теорему о корреляции, которая
обычно формулируется следующим образом:
\begin{equation}
    K_{\zeta}(\rho) = \ifft{\fft*{\xi} \cdot \fft{\xi}} = 
    \ifft{\abs{\fft{\xi}}^2}
\end{equation}

При помощи теоремы котельникова можно пока

\begin{equation}
    \label{eq:}
    \Delta x = \frac{\pi}{k}
\end{equation}





\end{document}

